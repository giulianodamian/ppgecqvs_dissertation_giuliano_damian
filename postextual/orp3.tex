\chapter{Aplicações da ORP 3} \label{apendice:orp3}
\section{Lista 1 - ORP3}\label{ch:orp3l1}
\begin{itemize}
    \item[1.] \textbf{\cite{halliday}} Um feixe de luz monocromática é absorvido por um filme fotográfico e fica registrado no filme. Um fóton é absorvido pelo filme se a energia do fóton for igual ou maior que a energia mínima de $0,6$ $eV$ necessária para desassociar uma molécula de AgBr do filme. (a) Qual é o maior comprimento de onda que pode ser registrado no filme? (b) A que região do espectro eletromagnético pertence esse comprimento de onda?

    \item[2.] \textbf{\cite{halliday}} (a) Qual é o momento, em $MeV/c$, de um fóton cuja energia é igual à energia de repouso de um elétron? Quais são (b) o comprimento de onda e (c) a frequência da radiação correspondente?

    \item[3.] \textbf{(\cite{halliday}/Adaptada)} A superfície do Sol se comporta aproximadamente como um corpo negro à temperatura de $5800$ $K$. (a) Calcule o comprimento de onda para o qual a radiância espectral da superfície do Sol é máxima e (b) indique em que região do espectro eletromagnético está esse comprimento deonda. (c) O universo se comporta aproximadamente como um corpo negro cuja radiação foi emitida quando os átomos se formaram pela primeira vez. Hoje em dia, o comprimento de onda para o qual a radiação desse corpo negro é máxima é $1,06$ $mm$ (um comprimento de onda que se encontra na faixa das micro-ondas). Qual é a temperatura atual do universo?

    \item[4.] \textbf{\cite{halliday}} No tubo de imagem de um velho aparelho de televisão, os elétrons são acelerados por uma diferença de potencial de $25,0$ $kV$. Qual é o comprimento de onda de de Broglie desses elétrons? (Não é necessário levar em conta efeitos relativísticos.)

    \item[5.] \textbf{\cite{halliday}} Mostre que o número de onda $k$ de uma partícula livre não relativística, de massa $m$, pode ser escrito na forma
    \begin{equation*}
        k = \frac{2 \pi \sqrt{2mK}}{h},
    \end{equation*}
    em que $K$ é a energia cinética da partícula.

    \item[6.] \textbf{\cite{halliday}} A indeterminação da posição de um elétron situado no eixo $x$ é $50$ $pm$, ou seja, um valor aproximadamente igual ao raio do átomo de hidrogênio. Qual é a menor indeterminação possível da componente $p$, do momento do elétron?

    \item[7.] \textbf{\cite{halliday}} (a) Um feixe de prótons de $5,0$ $eV$ incide em uma barreira de energia potencial de $6,0$ $eV$ de altura e $0,70$ $nm$ de largura, a uma taxa correspondente a uma corrente de $1000$ $A$. Quanto tempo é preciso esperar (em média) para que um próton atravesse a barreira? (b) QUanto tempo será preciso esperar, se o feixe contiver elétrons em vez de prótons?

    \item[8.] \textbf{\cite{halliday}} Um elétron está se movendo em uma região onde existe m potencial elétrico uniforme de $-200$ $V$ com uma enertgia (total) de $500$ $eV$. Determine (a) a energia cinética do elétron, em elétron-volts, (b) o momento do elétron, (c) a velocidade do elétron, (d) o coprimento de onda de de Broglie do elétron, (e) o número de onda do elétron.

    \item[9.] \textbf{\cite{halliday}} A resolução de um microscópio depende do comprimento de onda usado; o menor objeto que pode ser resolvido tem dimensões da oredem do comprimento de onda. Suponha que estamos interessados em "observar" o interior do átomo. Como um átomo tem um diâmetro da ordem de $100$ $pm$, isso significa que devemos ser capazes de resolver dimensões da ordem de $10$ $pm$. (a) Se um microscópio eletrônico for usado para este fim, qual deverá ser, no mínimo, a energia dos elétrons? (b) Se um microscópio ótico for usado, qual deverá ser, no mínimo,a energia dos fótons? (c) Qual dos dois microscópios parece ser mais prático? Por quê?

    \item[10.] \textbf{\cite{halliday}} Mostre que, se um fóton de energia $E$ for espalhado por um elétron livre em repouso,a energia cinética máxima do elétron espalhado será
    \begin{equation*}
        K_{max}= \frac{E^2}{E + mc^2/2}
    \end{equation*}
    
\end{itemize}
\newpage
\section{Lista 2 - ORP3} \label{ch:orp3l2}

\begin{itemize}
\item[1.] \textbf{\cite{Griffiths2011}} Demonstre que não há solução aceitável para a equação de Schrödinger (independente do tempo) para o poço quadrado infinito com $E = 0$ ou $E < 0$.

\item[2.] \textbf{\cite{Griffiths2011}} Uma partícula no potencial do oscilador harmônico está no estado
\begin{equation*}
\Psi (x,0) = A[3{\psi}_0 (x) + 4{\psi}_1(x)]
\end{equation*}
(a) Calcule A. 
\par
(b) Monte $\psi(x,t)$ e $|\psi(x,t)|^2$.
\par 
(c) Calcule $\langle x \rangle$ e $\langle p \rangle$.
\par 
(d) Ao medir a energia da partícula, que valores você poderá obter e quais as probabilidades de obter cada um deles?

\item[3.]\textbf{\cite{Griffiths2011}} Uma partícula livre tem a seguinte função de onda inicial:
\begin{equation*}
\Psi (x,0) = A e^{-a|x|},
\end{equation*}
em que $A$ e $a$ são constantes reais positivas.
\par 
(a) Normalize $\Psi(x,0)$.
\par 
(b) Calcule $\phi (k)$.
\par 
(c) Monte $\Psi (x,t)$ na forma integral.
\par 
(d) Discuta os casos limites ($a$ muito grande ou muito pequeno).

\end{itemize}
\newpage
\section{Lista 3 - ORP3} \label{ch:orp3l3}

\begin{itemize}
    \item[1.] \textbf{\cite{Griffiths2011}/Adaptada} 
    \begin{itemize}
        \item[a)] Resolva todas as \textbf{relações de comutação canônicas} para os componenetes dos operadores \textbf{r} e \textbf{p}: $[x,y]$, $[x,p_y]$, $[x,p_x]$, $[x,p_z]$, e assim por diante. \textit{Resposta}:
        \begin{center}
            $[r_i,p_j]=-[p_i,r_j]=i \hbar \delta_{ij}$, $[r_i,r_j]=[p_i,p_j]=0$,
        \end{center}
        em que os índices representam $x$, $y$ ou $z$, e $r_x=x$, $r_y=y$ e $r_z=z$.

        \item[b)] Confirme o teorema de Ehrenfes para três dimensões:
        \begin{center}
            $\frac{d}{dt} \langle \textbf{r} \rangle = \frac{1}{m} \langle \textbf{p} \rangle$ e $\frac{d}{dt}\langle \textbf{p} \rangle = \langle - \nabla V \rangle.$
        \end{center}

        \item[c)] Formule o princípio da incerteza de Heisenberg em três dimensões. \textit{Resposta}:
        \begin{center}
            $\sigma_x \sigma_{p_x} \geq \frac{\hbar}{2}$, $\sigma_y \sigma_{p_y} \geq \frac{\hbar}{2}$, $\sigma_z \sigma_{p_z} \geq \frac{\hbar}{2}$,
        \end{center}
        mas vamos dizer que não haja restrição em $\sigma_x \sigma_{p_y}$.
    \end{itemize}

    \item[2.] \textbf{\cite{Griffiths2011}/Adaptada} Utiliza as Equações abaixo para montar $Y^{0}_{0}$ e $Y^{1}_{2}$. Certifique-se de que eles estejam normalizados e sejam ortogonais.

    \begin{equation*}
        P^{m}_{l}(x) \equiv (1-x^2)^{|m|/2}\left( \frac{d}{dx} \right)^{|m|}P_{l}(x)
    \end{equation*}
    \begin{equation*}
        P_{l}(x) \equiv \frac{1}{2^l l!}\left( \frac{d}{dx} \right)^l (x^2-1)^l
    \end{equation*}
    \begin{equation*}
        Y^{m}_{l}(\theta,\phi) = \epsilon\sqrt{\frac{(2l+1)}{2 \pi}\frac{(l-|m|)!}{(l+|m|)!}}e^{im\phi}P^{m}_{l}( \cos \theta)
    \end{equation*} 
\end{itemize}