\chapter{ASA 1} \label{apendice:asa1}
\section{Aplicação 1} \label{ch:ASA1a1}
\begin{itemize}
    \item[Modelo A:] Calcule a velocidade média nos dois casos a seguir:
    \begin{itemize}
        \item[(a)] Você caminha $73,2$ $m$ a uma velocidade de $1,22$ $m/s$ e depois corre $73,2$ $m$ a $3,05$ $m/s$ em uma pista reta.
        \item[(b)] Você caminha $1,00$ $min$ com velocidade de $1,22$ $m/s$ e depois corre por $1,00$ $min$ a $3,05$ $m/s$ em uma pista reta.
    \end{itemize}
    \item[Modelo B:] Calcule a velocidade média nos dois casos a seguir:
    \begin{itemize}
        \item[(a)] Você caminha $75,2$ $m$ a uma velocidade de $1,26$ $m/s$ e depois corre $73,2$ $m$ a $3,05$ $m/s$ em uma pista reta.
        \item[(b)] Você caminha $1,00$ $min$ com velocidade de $1,22$ $m/s$ e depois corre por $1,00$ $min$ a $3,15$ $m/s$ em uma pista reta.
    \end{itemize}
    \item[Modelo C:] Calcule a velocidade média nos dois casos a seguir:
    \begin{itemize}
        \item[(a)] Você caminha $73,2$ $m$ a uma velocidade de $1,22$ $m/s$ e depois corre $73,2$ $m$ a $3,05$ $m/s$ em uma pista reta.
        \item[(b)] Você caminha $1,00$ $min$ com velocidade de $1,22$ $m/s$ e depois corre por $1,50$ $min$ a $3,25$ $m/s$ em uma pista reta.
    \end{itemize}
    \item[Modelo D:] Calcule a velocidade média nos dois casos a seguir:
    \begin{itemize}
        \item[(a)] Você caminha $79,2$ $m$ a uma velocidade de $1,36$ $m/s$ e depois corre $74,2$ $m$ a $3,05$ $m/s$ em uma pista reta.
        \item[(b)] Você caminha $1,00$ $min$ com velocidade de $1,22$ $m/s$ e depois corre por $2,00$ $min$ a $3,05$ $m/s$ em uma pista reta.
    \end{itemize}
\end{itemize}
\newpage
\section{Aplicação 2}\label{ch:ASA1a2}
\begin{itemize}
    \item[Modelo A:] Um trem de $150$ $m$ de comprimento se desloca com velocidade esalar constante de $57,6$ $km/h$. Esse trem atravessa um túnel e leva $50$ $s$ desde a entrada até a saída completa dentro dele. Qual é o comprimento total do túnel?
    
    \item[Modelo B:] Um trem de $200$ $m$ de comprimento se desloca com velocidade esalar constante de $50,4$ $km/h$. Esse trem atravessa um túnel e leva $40$ $s$ desde a entrada até a saída completa dentro dele. Qual é o comprimento total do túnel?
    
    \item[Modelo C:] Um trem de $100$ $m$ de comprimento se desloca com velocidade esalar constante de $61,2$ $km/h$. Esse trem atravessa um túnel e leva $120$ $s$ desde a entrada até a saída completa dentro dele. Qual é o comprimento total do túnel?
    
    \item[Modelo D:] Um trem de $75$ $m$ de comprimento se desloca com velocidade esalar constante de $57,6$ $km/h$. Esse trem atravessa um túnel e leva $240$ $s$ desde a entrada até a saída completa dentro dele. Qual é o comprimento total do túnel?
\end{itemize}
\newpage
\section{Aplicação 3} \label{ch:ASA1a3}
\begin{itemize}
    \item[Modelo A:] Em uma fase de testes para fazer entregas rápidas, um \textit{drone}, parado sobre local de entrega, deixa cair um pacote de uma altura de $150$ $m$ acima do solo. Desconsiderando os efeitos de resistência do ar e considerando $g=10$ $m/s^2$, determine (a) o tempo que o pacote leva para atingir o solo e (b) a velocidade do pacote ao atingir o solo.
    
    \item[Modelo B:] Em uma fase de testes para fazer entregas rápidas, um \textit{drone}, parado sobre local de entrega, deixa cair um pacote de uma altura de $160$ $m$ acima do solo. Desconsiderando os efeitos de resistência do ar e considerando $g=10$ $m/s^2$, determine (a) o tempo que o pacote leva para atingir o solo e (b) a velocidade do pacote ao atingir o solo.
    
    \item[Modelo C:] Em uma fase de testes para fazer entregas rápidas, um \textit{drone}, parado sobre local de entrega, deixa cair um pacote de uma altura de $181$ $m$ acima do solo. Desconsiderando os efeitos de resistência do ar e considerando $g=10$ $m/s^2$, determine (a) o tempo que o pacote leva para atingir o solo e (b) a velocidade do pacote ao atingir o solo.
    
    \item[Modelo D:] Em uma fase de testes para fazer entregas rápidas, um \textit{drone}, parado sobre local de entrega, deixa cair um pacote de uma altura de $175$ $m$ acima do solo. Desconsiderando os efeitos de resistência do ar e considerando $g=10$ $m/s^2$, determine (a) o tempo que o pacote leva para atingir o solo e (b) a velocidade do pacote ao atingir o solo.
    
\end{itemize}

