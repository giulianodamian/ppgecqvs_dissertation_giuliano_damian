\chapter{ASA 3} \label{apendice:asa3}
\section{Aplicação 1} \label{ch:ASA3a1}

\begin{itemize}
    \item[Modelo A:] Deseja-se utilizar um objeto metálico, inicialmente neutro, pelos processos de eletrização conhecidos, e obter uma quantidade de carga negativa $3,2$ $\mu C$. Será necessário acrescentar ou retirar elétrons? Quantos elétrons serão necessários? 
    
    \item[Modelo B:] Deseja-se utilizar um objeto metálico, inicialmente neutro, pelos processos de eletrização conhecidos, e obter uma quantidade de carga negativa $16,0$ $\mu C$. Será necessário acrescentar ou retirar elétrons? Quantos elétrons serão necessários? 
   
    \item[Modelo C:] Deseja-se utilizar um objeto metálico, inicialmente neutro, pelos processos de eletrização conhecidos, e obter uma quantidade de carga negativa $12,8$ $\mu C$. Será necessário acrescentar ou retirar elétrons? Quantos elétrons serão necessários? 
    
    \item[Modelo D:] Deseja-se utilizar um objeto metálico, inicialmente neutro, pelos processos de eletrização conhecidos, e obter uma quantidade de carga negativa $6,4$ $\mu C$. Será necessário acrescentar ou retirar elétrons? Quantos elétrons serão necessários? 

\end{itemize}
\newpage
\section{Aplicação 2} \label{ch:ASA3a2}
\begin{itemize}
    \item[Modelo A:] Duas cargas puntiformes, $Q_1 = 2$ $\mu C$ e $Q_2 = -6$ $\mu C$, estão fixas e separadas por uma distância de $600$ $nm$ no vácuo. Uma terceira carga $Q_3 = 3$ $\mu C$ é colocada no ponto médio do segmento que une as cargas $Q_1$ e $Q_2$. Qual o módulo da força elétrica resultante da carga $Q_3$? Qual a direção e o sentido desta força resultante?
    
    \item[Modelo B:] Duas cargas puntiformes, $Q_1 = 4$ $\mu C$ e $Q_2 = -5$ $\mu C$, estão fixas e separadas por uma distância de $600$ $nm$ no vácuo. Uma terceira carga $Q_3 = 1$ $\mu C$ é colocada no ponto médio do segmento que une as cargas $Q_1$ e $Q_2$. Qual o módulo da força elétrica resultante da carga $Q_3$? Qual a direção e o sentido desta força resultante?
    
    \item[Modelo C:] Duas cargas puntiformes, $Q_1 = 2$ $\mu C$ e $Q_2 = -6$ $\mu C$, estão fixas e separadas por uma distância de $800$ $nm$ no vácuo. Uma terceira carga $Q_3 = 2$ $\mu C$ é colocada no ponto médio do segmento que une as cargas $Q_1$ e $Q_2$. Qual o módulo da força elétrica resultante da carga $Q_3$? Qual a direção e o sentido desta força resultante?
    
    \item[Modelo D:] Duas cargas puntiformes, $Q_1 = 9$ $\mu C$ e $Q_2 = -6$ $\mu C$, estão fixas e separadas por uma distância de $300$ $nm$ no vácuo. Uma terceira carga $Q_3 = 8$ $\mu C$ é colocada no ponto médio do segmento que une as cargas $Q_1$ e $Q_2$. Qual o módulo da força elétrica resultante da carga $Q_3$? Qual a direção e o sentido desta força resultante?
    
\end{itemize}
\newpage
\section{Aplicação 3} \label{ch:ASA3a3}
\begin{itemize}
    \item[Modelo A:] Admita que a distância entre os eletrodos de um campo elétrico é de $20$ $cm$ e que a diferença de potencial efetia aplicada ao circuito é de $6$ $V$. Calcule (a) a intensidade do campo elétrico (em $V/m$) e (b) a força elétrica exercida por ese campo em uma partícula de carga $e$.
    
    \item[Modelo B:] Admita que a distância entre os eletrodos de um campo elétrico é de $50$ $cm$ e que a diferença de potencial efetia aplicada ao circuito é de $8$ $V$. Calcule (a) a intensidade do campo elétrico (em $V/m$) e (b) a força elétrica exercida por ese campo em uma partícula de carga $3e$.
    
    \item[Modelo C:] Admita que a distância entre os eletrodos de um campo elétrico é de $40$ $cm$ e que a diferença de potencial efetia aplicada ao circuito é de $6$ $V$. Calcule (a) a intensidade do campo elétrico (em $V/m$) e (b) a força elétrica exercida por ese campo em uma partícula de carga $3e$.
    
    \item[Modelo D:] Admita que a distância entre os eletrodos de um campo elétrico é de $20$ $cm$ e que a diferença de potencial efetia aplicada ao circuito é de $12$ $V$. Calcule (a) a intensidade do campo elétrico (em $V/m$) e (b) a força elétrica exercida por ese campo em uma partícula de carga $2e$.
    
\end{itemize}