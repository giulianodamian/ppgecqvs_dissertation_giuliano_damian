\chapter{ASA 2} \label{apendice:asa2}
\section{Aplicação 1} \label{ch:ASA2a1}

\begin{itemize}
    \item[Modelo A:] Em uma escala linear $X$, a água congela a $-125,0^{\circ}X$ e evapora a $375,0^{\circ}X$. Em uma escala linear de temperatura $Y$, a água congela a $-70,0^{\circ}Y$ e evapora a $-30,0^{\circ}Y$. Uma temperatura de $50,0^{\circ}Y$ corresponde a que temperatura na escala $X$?    

    \item[Modelo B:] Em uma escala linear $X$, a água congela a $-120,0^{\circ}X$ e evapora a $375,0^{\circ}X$. Em uma escala linear de temperatura $Y$, a água congela a $-70,0^{\circ}Y$ e evapora a $-30,0^{\circ}Y$. Uma temperatura de $50,5^{\circ}Y$ corresponde a que temperatura na escala $X$? 
   
    \item[Modelo C:] Em uma escala linear $X$, a água congela a $-125,0^{\circ}X$ e evapora a $380,0^{\circ}X$. Em uma escala linear de temperatura $Y$, a água congela a $-70,0^{\circ}Y$ e evapora a $-30,0^{\circ}Y$. Uma temperatura de $60,0^{\circ}Y$ corresponde a que temperatura na escala $X$? 
    
    \item[Modelo D:] Em uma escala linear $X$, a água congela a $-122,0^{\circ}X$ e evapora a $372,0^{\circ}X$. Em uma escala linear de temperatura $Y$, a água congela a $-70,0^{\circ}Y$ e evapora a $-30,0^{\circ}Y$. Uma temperatura de $60,5^{\circ}Y$ corresponde a que temperatura na escala $X$? 

    \end{itemize}
\newpage

\section{Aplicação 2} \label{ch:ASA2a2}
\begin{itemize}
    \item[Modelo A:] Calcule a quantidade de calor necessário para tranformar $200$ $g$ de gelo a $-5^{\circ}C$ em vapor de água a $107^{\circ}C$.
    
    \item[Modelo B:] Calcule a quantidade de calor necessário para tranformar $300$ $g$ de gelo a $-10^{\circ}C$ em vapor de água a $110^{\circ}C$.
    
    \item[Modelo C:] Calcule a quantidade de calor necessário para tranformar $100$ $g$ de gelo a $-15^{\circ}C$ em vapor de água a $120^{\circ}C$.
    
    \item[Modelo D:] Calcule a quantidade de calor necessário para tranformar $500$ $g$ de gelo a $-25^{\circ}C$ em vapor de água a $117^{\circ}C$.
    
\end{itemize}
Dados fornecidos em todos os modelos:
\begin{itemize}
    \item $c_{gelo} = 0,5 cal/g^{\circ}C$;
    \item $c_{agua} = 1 cal/g^{\circ}C$;
    \item $c_{gelo} = 0,40 cal/g^{\circ}C$;
    \item $L_{fusao} = 80 cal/g$;
    \item $L_{ebulicao} = 540 cal/g$.
\end{itemize}
\newpage

\section{Aplicação 3} \label{ch:ASA2a3}
\begin{itemize}
    \item[Modelo A:] Em uma fábrica usa-se uma barra de alumínio de $80$ $cm^2$ de seção reta e $20$ $cm$ de comprimento, para manter constante a temperatura da máquina em operação. Uma das extremidades da barra é colocada em contato com a máquina que opera à temperatura constante de $400^{\circ}C$, enquanto a outra extremidade está em contato com uma barra de gelo na sua temperatura de fusão. Sabendo-se que o calor latente de fusão do gelo é de $80$ $cal/g$, que o coeficiente de condutibilidade térmica do alumínio é de $0,5$ $cal/s \cdot cm \cdot ^{\circ}C$ e desprezando-se as trocas de calor do sistema máquina-gelo com o ambiente, calcule o tempo necessário para derreter $500$ $g$ de gelo.
    
    \item[Modelo B:] Em uma fábrica usa-se uma barra de alumínio de $60$ $cm^2$ de seção reta e $40$ $cm$ de comprimento, para manter constante a temperatura da máquina em operação. Uma das extremidades da barra é colocada em contato com a máquina que opera à temperatura constante de $400^{\circ}C$, enquanto a outra extremidade está em contato com uma barra de gelo na sua temperatura de fusão. Sabendo-se que o calor latente de fusão do gelo é de $80$ $cal/g$, que o coeficiente de condutibilidade térmica do alumínio é de $0,5$ $cal/s \cdot cm \cdot ^{\circ}C$ e desprezando-se as trocas de calor do sistema máquina-gelo com o ambiente, calcule o tempo necessário para derreter $100$ $g$ de gelo.
    
    \item[Modelo C:] Em uma fábrica usa-se uma barra de alumínio de $80$ $cm^2$ de seção reta e $30$ $cm$ de comprimento, para manter constante a temperatura da máquina em operação. Uma das extremidades da barra é colocada em contato com a máquina que opera à temperatura constante de $400^{\circ}C$, enquanto a outra extremidade está em contato com uma barra de gelo na sua temperatura de fusão. Sabendo-se que o calor latente de fusão do gelo é de $80$ $cal/g$, que o coeficiente de condutibilidade térmica do alumínio é de $0,5$ $cal/s \cdot cm \cdot ^{\circ}C$ e desprezando-se as trocas de calor do sistema máquina-gelo com o ambiente, calcule o tempo necessário para derreter $200$ $g$ de gelo.
    
    \item[Modelo D:] Em uma fábrica usa-se uma barra de alumínio de $100$ $cm^2$ de seção reta e $10$ $cm$ de comprimento, para manter constante a temperatura da máquina em operação. Uma das extremidades da barra é colocada em contato com a máquina que opera à temperatura constante de $400^{\circ}C$, enquanto a outra extremidade está em contato com uma barra de gelo na sua temperatura de fusão. Sabendo-se que o calor latente de fusão do gelo é de $80$ $cal/g$, que o coeficiente de condutibilidade térmica do alumínio é de $0,5$ $cal/s \cdot cm \cdot ^{\circ}C$ e desprezando-se as trocas de calor do sistema máquina-gelo com o ambiente, calcule o tempo necessário para derreter $1000$ $g$ de gelo.
    
\end{itemize}