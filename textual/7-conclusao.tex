\chapter{Considerações Finais} \label{ch:conclusao}

Este trabalho foi desenvolvido no âmbito do EDR, uma abordagem de pesquisa que busca encontrar soluções práticas para problemas reais no contexto da educação. Nessa perspectiva, nosso objetivo foi analisar a competência de RP em Física, utilizando a metodologia de \textit{Scaffolding} como uma ferramenta de apoio ao processo de aprendizagem dos alunos. Essa metodologia oferece suporte estruturado e gradual aos estudantes, auxiliando-os no desenvolvimento de habilidades necessárias para resolver problemas de forma autônoma e eficiente.

A ideia inicial do projeto, concebida em meados de 2019, precisou ser adaptada para uma nova realidade trazida pela Pandemia de coronavírus. Esse processo resultou na implementação da ORP1 em uma turma de Engenharia Civil, utilizando o formato de aulas remotas.

Durante todo o processo de coleta de dados, enfrentamos uma série de desafios que merecem destaque. Tanto durante a realização da ORP1 quanto nas aplicações subsequentes, a influência da pandemia foi manifesta em nossas atividades.

Durante a ORP1, vivenciamos o auge da pandemia de forma direta, o que nos obrigou a adotar um ambiente de ensino remoto. Nas ORP2, ORP3 e nas ASA, pudemos observar que os períodos de aulas remotas e isolamento social tiveram um impacto direto no desenvolvimento dos alunos, especialmente aqueles do nível médio, nas ASA.

É fundamental salientar que este estudo foi conduzido em um período de tempo singular, no qual os resultados obtidos possivelmente teriam sido diferentes se não tivéssemos enfrentado uma situação tão complexa de saúde pública em escala global.

Durante a primeira aplicação da metodologia de \textit{Scaffolding}, diversos aspectos prejudicaram uma análise mais completa e precisa. Entre esses aspectos, destacam-se a perda das respostas dos questionários dos alunos, o impacto da Pandemia que afetou tanto o formato de aulas remotas quanto o aspecto psicológico dos estudantes, e a identificação de possíveis casos de plágio nas resoluções dos alunos. Esses fatores dificultaram a obtenção de informações valiosas e confiáveis para uma análise mais aprofundada da eficácia da metodologia em questão.

A perda das respostas dos questionários dos alunos representou uma limitação significativa para a compreensão de suas percepções e opiniões sobre a experiência com a metodologia de \textit{Scaffolding}. Sem esses dados, tornou-se mais desafiador entender como os alunos reagiram ao processo de ensino e aprendizagem, e quais pontos específicos poderiam ser aprimorados ou adaptados para melhor atender suas necessidades.

Além disso, a situação da Pandemia trouxe impactos significativos para o ambiente educacional e para o bem-estar dos estudantes. A transição abrupta para o ensino a distância pode ter gerado desafios adicionais para o envolvimento e a motivação dos alunos, afetando suas experiências de aprendizagem e a interação com a metodologia proposta.

Outro aspecto que exigiu atenção foi a identificação de semelhanças suspeitas nas resoluções dos alunos. Essas similaridades indicavam a possibilidade de plágio, o que trouxe desafios para a análise da autenticidade das respostas e a confiabilidade dos resultados obtidos. Esse foi um fator que precisou ser considerado para garantir a integridade acadêmica do processo de avaliação. 

Diante dos desafios e limitações encontrados na primeira aplicação da metodologia de \textit{Scaffolding}, foi necessário realizar um \textit{redesign} do projeto para continuar a análise de forma mais apropriada. Esse novo \textit{design}, alinhado com os princípios do EDR, resultou em duas novas aplicações: ORP2 e ORP3. 

As novas aplicações, ORP2 e ORP3, foram realizadas simultaneamente em turmas de cursos de Matemática e Física. Além dos objetivos gerais a serem analisados, também surgiu o interesse em investigar as possíveis diferenças entre essas turmas, uma composta por estudantes novatos e outra por especialistas em Física, durante o processo de RP.

Devido ao baixo número de alunos matriculados nas disciplinas da ORP2 e ORP3, a análise comparativa entre as turmas e outros objetivos não puderam ser completamente elucidados. Diante disso, optou-se por um novo \textit{redesign} da metodologia, direcionado para turmas do Ensino Médio. 

A metodologia ASA foi implementada simultaneamente em três turmas distintas do Ensino Médio: uma no 1º ano, outra no 2º ano e uma terceira no 3º ano de uma Escola Particular e Confessional localizada no interior do Estado do Rio Grande do Sul. Apesar das turmas possuírem conhecimentos semelhantes em relação à Física, as particularidades e características individuais de cada grupo de alunos mostraram-se bastante distintas. Essas diferenças podem influenciar na forma como os alunos abordam e enfrentam os desafios das aplicações da ASA.

A turma do 1º ano demonstrou interesse em participar da metodologia da ASA. No entanto, seus resultados e relatos do Professor indicam que a dedicação aos estudos foi insuficiente. Muitos alunos pareciam depender demasiadamente das dicas fornecidas ao longo das aplicações, na esperança de obterem uma boa nota sem um esforço mais aprofundado de estudo. Essa atitude pode ter afetado o desempenho geral da turma na ASA.

A turma do 2º ano apresentou uma divisão significativa em relação à participação nas aplicações da ASA. Muitos alunos enfrentavam dificuldades, mas não solicitavam ajuda, o que pode indicar uma falta de engajamento nos estudos fora de sala de aula. Como resultado, os desempenhos individuais não foram considerados satisfatórios pela escola. Essa falta de engajamento pode ter influenciado diretamente nos resultados alcançados .

A turma do 3º ano apresentou uma peculiaridade em relação à sua percepção sobre a ASA ao longo do processo. Inicialmente, eles não entenderam a função dessas atividades no seu processo de aprendizagem, o que levou a uma visão equivocada sobre a finalidade da ASA. Como estavam preocupados com as provas de vestibular e ENEM que enfrentariam no final do ano, essa percepção distorcida pode ter influenciado na forma como abordaram e interagiram com as aplicações. À medida que a compreensão sobre a importância da ASA se desenvolveu, suas atitudes podem ter se alterado, refletindo-se nos resultados obtidos ao final do processo.

Apesar de termos realizado uma análise detalhada dos objetivos em cada aplicação das ORP e das ASA no Capítulo \ref{ch:resanddisc}, torna-se necessário uma síntese geral desses objetivos. Com esse propósito, apresentamos na Tabela \ref{tab:objetivosResp} nossa análise das respostas dos objetivos específicos deste trabalho, abrangendo todas as aplicações realizadas nos diferentes \textit{designs}.

\begin{table}[ht]
\centering
\caption{Análise dos Objetivos Específicos}
\label{tab:objetivosResp}
\begin{tabular}{l p{10cm}}
\hline
\textbf{Objetivo \ref{item:a}} & Os alunos que obtiveram sucesso na resolução dos problemas propostos seguiram passos similares aos de \citeonline{Pozo2009}. \\
\hline
\textbf{Objetivo \ref{item:b}} & As etapas em que as dicas são fornecidas variam consideravelmente, mas são frequentes nas dúvidas relacionadas a Conteúdo e Matemática. \\
\hline
\textbf{Objetivo \ref{item:c}} & A evolução temporal é um processo individual de cada estudante, sendo necessário analisá-la dentro de um estudo de caso para obter uma compreensão mais detalhada e específica. Dessa forma, este objetivo não foi alcançado. \\
\hline
\textbf{Objetivo \ref{item:d}} & As formas mais eficientes de dicas para a compreensão dos alunos podem variar de acordo com o tipo de dica fornecida. No entanto, dicas transmitidas de forma mais objetiva parecem agradar mais aos estudantes. \\
\hline
\textbf{Objetivo \ref{item:e}} & O interesse dos alunos na RP depende da compreensão da metodologia. Quando entendem a proposta, se engajam; caso contrário, a veem como "perda de tempo". \\
\hline
\end{tabular}
\caption*{Fonte: Criação do Autor.}
\end{table}

É relevante salientar que a abordagem de RP proposta por \citeonline{Pozo2009}, conforme abordado no objetivo \ref{item:a}, nunca foi instruída diretamente aos alunos. Eles a empregam de forma natural, possivelmente influenciados pela forma como outras pessoas, como pais ou professores, resolvem problemas. Essa influência pode levá-los a adaptar seu processo de resolução, aproximando-o do que é considerado mais adequado.

Adicionalmente, constatou-se que vários dos objetivos traçados inicialmente exibiram variações conforme as circunstâncias e particularidades de cada aluno e turma. Nesse sentido, sugere-se que estudos de caso mais abrangentes e prolongados sejam conduzidos no futuro, permitindo uma investigação mais aprofundada das questões relacionadas ao processo de RP em diferentes contextos educacionais.

Com base nas informações e experiências vivenciadas ao longo de todo o processo, desde a ORP1 até a ASA3, evidencia-se que a metodologia de \textit{Scaffolding} demonstra contribuições significativas para o desenvolvimento da competência de RP em Física. Contudo, existe espaço para aprimorar as estratégias de aplicação, a fim de obter resultados mais abrangentes e abordar uma maior variedade de grupos estudados. 

Recomendamos que o próximo \textit{redesign} seja planejado com um acompanhamento mais prolongado de um indivíduo ou grupo. No contexto de uma turma do Ensino Médio, seria vantajoso acompanhar o progresso da mesma ao longo de um ano letivo, pois um período de tempo mais extenso provavelmente revelaria uma evolução mais significativa no desempenho dos alunos ao final do processo. Isso possibilitaria uma análise mais aprofundada das mudanças ao longo do tempo, identificando padrões de desenvolvimento e áreas de melhoria na competência de RP em Física.

No contexto do Ensino Superior, o acompanhamento individual de um aluno se torna mais viável, devido à natureza das disciplinas, que geralmente são oferecidas de forma individualizada, e à variação das turmas entre diferentes disciplinas. Dessa maneira, torna-se possível acompanhar o estudante por mais de um semestre, permitindo uma análise mais aprofundada e prolongada de seu processo de evolução.

Acredita-se que, no Ensino Superior, o diploma de graduação possa servir como um fator motivacional relevante para os estudantes, uma vez que a conclusão do curso é um objetivo claro e almejado para a maioria deles. No entanto, no Ensino Médio, a motivação dos alunos muitas vezes está mais voltada para a obtenção de boas notas do que para o próprio aprendizado em si. 

Nesse contexto, para promover uma melhor interação dos alunos do Ensino Médio com o processo de aplicação da metodologia \textit{Scaffolding}, sugere-se desvincular a aplicação da metodologia de uma avaliação puramente baseada em notas. Em vez disso, é importante explorar outras abordagens para atrair os estudantes para o processo de RP e, consequentemente, para a utilização da metodologia.