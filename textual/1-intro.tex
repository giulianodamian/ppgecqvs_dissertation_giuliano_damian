\chapter{Introdução}\label{ch:intro}

\section{Estrutura da Dissertação}

O presente trabalho é organizado em sete capítulos. No Capítulo \ref{ch:intro}, são expostas as fundamentações que motivaram a elaboração desta dissertação, além dos objetivos delineados para a pesquisa.

No Capítulo \ref{ch:edr}, abordamos os conceitos fundamentais da nossa abordagem de pesquisa, o \textit{Educational Design Research} (EDR). Nesse capítulo, enfatizamos os aspectos essenciais dessa concepção, bem como a perspectiva adotada pelo nosso grupo de pesquisa - Métodos e processos de Ensino e Aprendizagem (MPEAC).

No Capítulo \ref{ch:rp}, serão discutidos os principais aspectos da Resolução de Problemas (RP), que é o foco central desta dissertação. Nessa seção, serão abordados os fundamentos teóricos e práticos relacionados à abordagem de RP e sua relevância para o desenvolvimento da pesquisa.

No Capítulo \ref{ch:schaffolding}, será realizada uma análise detalhada da metodologia de \textit{Scaffolding}, que serve como base para esta dissertação. É importante ressaltar que os conteúdos abordados nos Capítulos \ref{ch:edr}, \ref{ch:rp} e \ref{ch:schaffolding} constituem os três pilares fundamentais para a implementação desta pesquisa.

No Capítulo \ref{ch:apl}, discutiremos como foram concebidas as Oficinas de Resolução de Problemas (ORP) e as Atividades de Sala de Aula (ASA). Todo esse processo foi pensado e desenvolvido visando analisar a competência de RP em Física e explorar a metodologia de \textit{Scaffolding} como uma ferramenta de apoio ao aprendizado dos alunos.

No Capítulo \ref{ch:resanddisc}, procedemos com uma análise detalhada e individual de cada ORP e ASA. Dessa forma, visando correlacionar os resultados obtidos com os objetivos específicos deste estudo. 

No Capítulo \ref{ch:conclusao}, apresentamos nossas considerações finais sobre a metodologia de \textit{Scaffolding}. Neste capítulo, abordamos a resposta ao objetivo geral da pesquisa, bem como fornecemos uma análise mais ampla dos objetivos específicos. Além disso, oferecemos sugestões para futuras pesquisas e possíveis desdobramentos do estudo.

\section{Justificativa}

 Após diálogos no grupo MPEAC, foi constatado que os alunos apresentam dificuldades na resolução de problemas de Física, persistindo desde o Ensino Fundamental até a primeira metade do ensino superior. Essas dificuldades se mostram como um desafio a ser superado no processo de aprendizagem.
 
A dificuldade na resolução de problemas apresenta uma preocupação relevante para a formação de futuros profissionais, uma vez que interfere na habilidade de interpretar situações-problema e elaborar estratégias de resolução. Durante o desenvolvimento deste trabalho, será demonstrado que a existência de lacunas em outras áreas de ensino, principalmente em Matemática, contribui para a intensificação das dificuldades dos alunos. 

A resolução de problemas é uma preocupação que engloba tanto os resultados quanto o processo de desenvolvimento. É fundamental considerá-la como uma etapa essencial no processo de ensino-aprendizagem de Física, uma vez que por meio dela os alunos tem seus conhecimentos conceituais e atitudinais postos à prova e são desafiados a aplicá-los de forma prática.

\citeonline{lurdes2018}, \citeonline{silva2019} e \citeonline{oliveira2017} discutem os desafios da Resolução de Problemas em Física. Além disso, de forma geral, concordam que as dificuldades no processo de resolução, apesar de ser algo amplamente estudado \cite{oliveira2017}, persistem até os dias atuais.


Com efeito, a resolução de um problema requer uma motivação inicial, seguida por uma análise cuidadosa da situação-problema e, por fim, a elaboração e execução de um plano de resolução. Essa sequência de passos é fundamental para o processo efetivo de solução de problemas em qualquer contexto, inclusive no ensino e aprendizagem de Física.

Frequentemente, os estudantes encaram as dificuldades nos problemas como obstáculos insuperáveis, o que pode desmotivá-los a prosseguir no processo de resolução. Nesse contexto, acreditamos que é papel do professor auxiliar o aluno durante essa etapa crucial de aprendizagem, oferecendo suporte e orientação para ajudá-los a superar tais barreiras e aprimorar suas habilidades de resolução de problemas.

Poderíamos mencionar diversas abordagens que permitem ao professor auxiliar os alunos nos processos de resolução de problemas, mas neste trabalho, focalizaremos a metodologia de \textit{Scaffolding}. Em um estudo de \citeonline{wood1976} explorou esse processo de tutoria com grupos de crianças, buscando desenvolver, entre outros efeitos, suas habilidades motoras. Neste contexto, nosso objetivo é investigar essa abordagem como forma de auxiliar os estudantes na resolução de problemas de Física.

O método de \textit{Scaffolding} foi selecionado por sua aplicabilidade em diversos contextos e por oferecer diferentes formas de utilização, além de possuir uma implementação simples e facilmente compreensível pelos alunos. Essa abordagem consiste no fornecimento de "andaimes" (tradução do inglês "\textit{Scaffolds}"), que servem de suporte para o desenvolvimento do processo de resolução de problemas.

De fato, assim como os andaimes temporários empregados em construções, as dicas utilizadas na metodologia de \textit{Scaffolding} não são permanentes. Elas têm a função de auxiliar o aluno no processo de resolução de problemas, sendo gradativamente removidas à medida que o estudante adquire as habilidades e conhecimentos necessários para solucionar o problema de forma autônoma e completa. O propósito é promover a capacidade do aluno de aplicar o aprendizado adquirido de maneira efetiva e sustentada.

Com base nessa perspectiva, acreditamos que um \textit{Scaffolding} bem estruturado e aplicado adequadamente possa auxiliar o aluno a superar dificuldades e avançar em sua resolução de problemas. Ao fornecer um suporte estratégico e personalizado, o \textit{Scaffolding} pode capacitar os estudantes a enfrentarem desafios com maior confiança e sucesso, estimulando o desenvolvimento de suas habilidades cognitivas e metacognitivas.

Nossa abordagem de pesquisa segue o gênero \textit{Educational Design Research}, que se caracteriza pela aplicação contínua de investigações educacionais, permitindo, se necessário, um \textit{redesign} e uma subsequente reaplicação, com o objetivo de buscar aprimorar os resultados alcançados. Essa metodologia permite uma abordagem iterativa e adaptativa, buscando constantemente a melhoria do processo de ensino-aprendizagem e o desenvolvimento de estratégias mais efetivas para a resolução de problemas em Física.

Essa dissertação se apoia em três pilares fundamentais: a Resolução de Problemas em Física, a abordagem de pesquisa EDR e a metodologia de \textit{Scaffolding}. Através da aplicação do método de \textit{Scaffolding} para a Resolução de Problemas de Física, sob a perspectiva do EDR, buscamos compreender a efetividade dessa abordagem na promoção do desenvolvimento da competência de resolução de problemas dos alunos. Ao adotar o EDR, visamos aprimorar a metodologia de ensino e aprendizagem, identificando possíveis ajustes e otimizações ao longo do processo, com o objetivo de aprimorar os resultados e proporcionar uma melhor experiência de aprendizado para os estudantes.

\section{Objetivos}

Com o objetivo de orientar esta investigação, foram formulados um objetivo geral e cinco objetivos secundários para esta dissertação.

\textbf{Objetivo geral}: Verificar se a metodologia do \textit{Scaffolding} consegue contribuir para o desenvolvimento da competência da resolução de problemas em Física.

 
\textbf{Objetivos Secundários:}

\begin{enumerate}[label=\alph*)]
    \item \label{item:a} Compreender como os alunos resolvem os problemas de Física; 
    \item \label{item:b} Quais etapas da resolução de problemas demandam mais auxílio; 
    \item  \label{item:c} Verificar a evolução temporal das resoluções dos problemas; 
    \item \label{item:d} Quais formas de ajuda são mais eficientes para a compreensão dos alunos;
    \item \label{item:e} O interesse na resolução dos problemas se mantém ao longo do semestre.
\end{enumerate}