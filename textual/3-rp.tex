\chapter{Resolução de Problemas}\label{ch:rp}

Neste trabalho, utilizamos a Resolução de Problemas como estratégia didática para promover o ensino-aprendizagem. Dessa maneira, é de fundamental importância fazermos uma discussão acerca da referida temática (RP). Nesse contexto, uma vez que a definição de problema é polissêmica (ou seja, pode ter diferentes definições a depender do contexto), realizaremos uma discussão no sentido de definir o significado de problema no âmbito do presente trabalho. 

De acordo com \citeonline[p.15]{Echeverria1998} e \citeonline{Lester1983}: que traz uma definição mais tradicional do termo problema poderia ser identificado como “uma situação que um indivíduo ou um grupo quer ou precisa resolver e para a qual não dispõe de um caminho rápido e direto que leve à solução". 

Além da noção de problema, precisamos entender o conceito de exercício que poderia ser identificado como uma situação em que um indivíduo ou grupo de indivíduos quer ou precisa resolver e, para isso, dispõe de um caminho rápido e direto que leva à solução. Ou seja, enquanto um problema apresenta uma novidade do ponto de vista de sua solução, o exercício, por sua vez, é uma repetição de habilidades ou técnicas sobre aprendidas \cite{Echeverria1998}. Como veremos, essa diferença nos conceitos de problema e exercício pode parecer confusa em um primeiro momento, mas compreender essa relação é essencial no emprego da RP como estratégia didática. 

\section{A IMPORTÂNCIA DA DISCUSSÃO ENTRE PROBLEMA E EXERCÍCIO NA PRÁTICA PEDAGÓGICA}

Segundo o que discutimos anteriormente, temos diferenças nos conceitos de problema e exercício, e é muito importante (do ponto de vista pedagógico) saber diferenciá-los. O exercício possui um algoritmo de solução, onde se treina a destreza do aluno, que faz com que a aprendizagem seja mecânica. Tão importante quanto entender esses conceitos é compreender que ambos constituem diferentes tipos de estratégias didáticas e que possuem finalidades didáticas diferentes. 

Apesar de termos uma classificação para os termos problema e exercício, não temos como definir, de forma geral, se determinada atividade é um ou outro. Isso ocorre devido ao fato de que a classificação em problema ou exercício depende de quem vai tentar solucioná-lo. Ou seja, uma determinada atividade pode ser um exercício para o professor, mas um problema para o aluno. Além disso, um problema que seja repetido várias vezes por uma pessoa pode passar a ser um simples exercício, pois ela já identificou os meios de solução e dominou o método. Ou seja, essa mudança da definição de uma atividade como problema ou exercício depende de cada pessoa. 

Assim, como conclui \citeonline[p.17]{Echeverria1998}: “a solução de problemas e a realização de exercícios constituem um continuum educacional cujos limites nem sempre são fáceis de estabelecer”. 

\section{CONTEÚDOS NA RESOLUÇÃO DE PROBLEMAS}

A RP constitui uma importante estratégia didática para o entendimento dos conteúdos, tanto na educação básica quanto na educação superior. Precisamos compreender que quando ensinamos os alunos a resolver problemas estamos, também, ensinando-os a desenvolver certos conteúdos, sejam eles atitudinais, conceituais ou procedimentais, e estratégias que poderão ser reproduzíveis e aplicáveis em novos contextos. Essa característica faz com que a RP seja aplicável na maioria das áreas de conhecimento. 

Quando pensamos em RP, tradicionalmente pensamos no ensino de Matemática. Porém, podemos trabalhar a RP em diferentes contextos, como nas Ciências Naturais ou até mesmo nas Ciências Sociais. Precisamos, portanto, estar cientes da existência dos diversos tipos de classificação para os problemas, seja em função da área de conhecimento, do conteúdo, dos tipos de operações envolvidas, ou de outras características \cite{Echeverria1998}. Essas diferenças existem por causa da forma como se estrutura os conceitos nas diferentes áreas e do tipo de conhecimento que é exigido para resolvê-los. Traremos algumas dessas classificações nas próximas seções. 

Para que seja possível ensinar a resolver um problema é necessário que o professor tenha consciência da complexidade dos passos para resolvê-lo. Essa afirmação pode parecer óbvia, porém, quando se domina determinada técnica ou habilidade, é comum que um professor não perceba as dificuldades mais “simples” que o estudante possui. \citeonline{Polya2006}, em seus estudos centrados na resolução de problemas em Matemática, criou uma sequência de etapas que serviriam para esse fim, demonstrados na Tabela \ref{tab:passos}. 

Podemos então resumir as observações de \cite{Polya2006} e dizer que para resolvermos um problema precisamos de quatro etapas: Compreensão do problema, concepção de um plano, execução de um plano e análise da solução \cite{Echeverria1998}. O conteúdo dessas etapas pode variar de acordo com o tipo, a área ou a peculiaridade do problema. Mas essa sequência algorítmica serve muito bem aos nossos propósitos. 

O primeiro passo para a solução de problemas é, portanto, a compreensão do próprio problema. Afinal, para que possamos classificar uma determinada atividade como problema, devemos estar conscientes de que estamos frente uma situação nova, que houve alguma mudança em relação ao caso anterior, ou, ainda, que temos explicações insuficientes para a realização da tarefa \cite{Echeverria1998}. 

Resumidamente, os primeiros passos para a resolução de um problema é a percepção do que já é conhecido pela pessoa e o entendimento das novas informações disponíveis na questão. Se alguma parte desse passo não puder ser desenvolvida, acabamos por ter uma dificuldade a mais para o processo de resolução. Afinal, se não conseguirmos interpretar as informações que nos sãos fornecidas, o avanço para a segunda etapa se torna inviável. Dessa forma, pode ser que seja necessário rever o conteúdo do problema e reestudar os tópicos pertinentes à questão, para que se possa desenvolver uma base de conhecimento suficiente para resolver o problema. 

\begin{comment}
    \begin{center}
\parbox{\textwidth}{%
\captionof{table}{Passos necessária para resolver um problema.}
\label{tab:passos}
\begin{mdframed}
\textbf{Compreender o problema}

(a) Qual é a incógnita? Quais são os dados? 
(b) Qual é a condição? A condição é suficiente para determinar a incógnita? É suficiente? Redundante? Contraditória? 

\textbf{Conceber um plano}

Já encontrou um problema semelhante? Ou já viu o mesmo problema proposto de maneira um pouco diferente? 
Conhece um problema relacionado com este? Conhece algum teorema que possa lhe ser útil? Olhe a incógnita com atenção e tente lembrar um problema que lhe seja familiar ou que tenha a mesma incógnita, ou uma incógnita similar. 
Este é um problema relacionado com o seu e que já foi resolvido. Você poderia utilizá-lo? Poderia usar o seu resultado? Poderia empregar o seu método? Considera que seria necessário introduzir algum elemento auxiliar para poder utilizá-lo? 
Poderia enunciar o problema de outra forma? Poderia apresentá-lo de forma diferente novamente? Refira-se às definições.
Se não pode resolver o problema proposto, tente resolver primeiro algum problema semelhante. Poderia imaginar um problema análogo um pouco mais acessível? Um problema mais geral? Um problema mais específico? Pode resolver uma parte do problema? Considere somente uma parte da condição; descarte a outra parte. Em que medida a incógnita fica agora determinado? De que forma pode variar? Você pode deduzir dos dados algum elemento útil? Pode pensar em outros dados apropriados para determinar à incógnita? Pode mudar a incógnita ou os dados, ou ambos, se necessário, de tal forma que a nova incógnita e os novos dados estejam mais próximos entre si?Empregou todos os dados? Empregou toda a condição? Considerou todas as noções essenciais concernentes ao problema? 

\textbf{Execução do plano }

(a) Ao executar o seu plano de resolução, comprove cada um dos passos. 
(b) Pode ver claramente que o passo é o correto? Pode demonstrá-lo? 

\textbf{Visão retrospectiva}

Pode verificar o resultado? Pode verificar o raciocínio?
Pode obter o resultado de forma diferente? Pode vê-lo com apenas uma olhada? Você pode empregar o resultado ou o método em outro problema? 
\end{mdframed}
\caption*{Fonte: Adaptado de \citeonline{Echeverria1998,Polya2006}.} %
}
\end{center}
\end{comment}

\begin{table}[ht]
\caption{Passos necessários para resolver um problema.}
\label{tab:passos}
\begin{tabularx}{\textwidth}{@{}X@{}}
\toprule
\textbf{Compreender o problema} \\
(a) Qual é a incógnita? Quais são os dados? \\
(b) Qual é a condição? A condição é suficiente para determinar a incógnita? É suficiente? Redundante? Contraditória? \\
\midrule
\textbf{Conceber um plano} \\
Já encontrou um problema semelhante? Ou já viu o mesmo problema proposto de maneira um pouco diferente? \\
Conhece um problema relacionado com este? Conhece algum teorema que possa lhe ser útil? Olhe a incógnita com atenção e tente lembrar um problema que lhe seja familiar ou que tenha a mesma incógnita, ou uma incógnita similar. \\
Este é um problema relacionado com o seu e que já foi resolvido. Você poderia utilizá-lo? Poderia usar o seu resultado? Poderia empregar o seu método? Considera que seria necessário introduzir algum elemento auxiliar para poder utilizá-lo? \\
Poderia enunciar o problema de outra forma? Poderia apresentá-lo de forma diferente novamente? Refira-se às definições. \\
Se não pode resolver o problema proposto, tente resolver primeiro algum problema semelhante. Poderia imaginar um problema análogo um pouco mais acessível? Um problema mais geral? Um problema mais específico? Pode resolver uma parte do problema? Considere somente uma parte da condição; descarte a outra parte. Em que medida a incógnita fica agora determinada? De que forma pode variar? Você pode deduzir dos dados algum elemento útil? Pode pensar em outros dados apropriados para determinar à incógnita? Pode mudar a incógnita ou os dados, ou ambos, se necessário, de tal forma que a nova incógnita e os novos dados estejam mais próximos entre si? \\
Empregou todos os dados? Empregou toda a condição? Considerou todas as noções essenciais concernentes ao problema? \\
\midrule
\textbf{Execução do plano} \\
(a) Ao executar o seu plano de resolução, comprove cada um dos passos. \\
(b) Pode ver claramente que o passo é o correto? Pode demonstrá-lo? \\
\midrule
\textbf{Visão retrospectiva} \\
Pode verificar o resultado? Pode verificar o raciocínio? \\
Pode obter o resultado de forma diferente? Pode vê-lo com apenas uma olhada? Você pode empregar o resultado ou o método em outro problema? \\
\bottomrule
\end{tabularx}
\caption*{Fonte: Adaptado de \citeonline{Echeverria1998,Polya2006}.}
\end{table}



A segunda etapa na resolução de problemas desenvolvida por \cite{Polya2006} é a concepção de um plano de ação. Ou seja, após identificarmos quais as informações previamente conhecidas e as novas informações, é necessário desenvolver uma estratégia para aplicar esses conhecimentos em prol da resolução. Na Tabela \ref{tab:passos} podemos ver diversos tipos de questionamentos que auxiliam na hora de desenvolver um plano. 

Na terceira etapa da resolução de problemas é necessário pôr em prática o plano desenvolvido. Após isso, na quarta etapa, é verificado se o resultado condiz com a pergunta. É de suma importância perceber que as quatro etapas de \citeonline{Polya2006} para a resolução de problemas formam uma sequência bem definida para a resolução. Se uma das etapas não for corretamente executada, acarretará num erro sistemático no restante do processo. Além disso, devemos dar uma atenção especial para o quarto passo. Pois sem ele não seria possível saber se todas as etapas anteriores foram corretamente trabalhadas. 

Podemos citar o fato de que o processo de resolução de problemas de \citeonline{Polya2006} é extremamente linear. Como fora discutido anteriormente, se uma etapa não puder ser concluída, não poderemos seguir adiante com a próxima. Entretanto, a solução de problemas não segue sempre uma sequência linear como estamos descrevendo \cite{Echeverria1998}. Quando introduzimos uma questão para resolver uma etapa do problema, por exemplo, “Considera que seria necessário introduzir algum elemento auxiliar para poder resolver o problema”, acabamos gerando um novo problema que requer que os mesmos quatro passos sejam seguidos. Dessa forma, acabamos por entrar em um loop de RP. 

Ao tentar resolver um problema, a forma de abordar e desenvolver seus passos vai variar de pessoa para pessoa, de acordo com seus conhecimentos prévios e motivações na hora da resolução. Ou seja, mesmo tendo uma “receita” para seguir no processo da RP, cada pessoa pode seguir por caminhos diferentes, utilizando de habilidades e estratégias distintas. Essa diferença se torna mais evidente entre especialistas e principiantes da área do problema em questão. 
\section{ESPECIALISTAS E PRINCIPIANTES NA RP}
Recapitulando, na hora de resolver um determinado problema, pessoas diferentes podem resolver de modos diferentes, mesmo seguindo os passos para RP descritos por \citeonline{Polya2006}. Isso se deve ao fato de que cada uma das pessoas possui uma bagagem de conhecimentos e experiências diferentes, o que pode facilitar ou dificultar o processo de RP. Como critério de classificação, podemos separar as pessoas em dois grupos distintos: os especialistas e os principiantes. 

Antes de qualquer coisa, é preciso ressaltar que a separação e análise sobre solução de problemas por especialistas e principiantes está fundamentada no princípio de que as habilidades e estratégias de solução de problemas são específicas de um determinado domínio de conhecimento, e, por isso, dificilmente se tornam transferíveis para outras áreas de conhecimento \cite{Echeverria1998}. Assim sendo, uma pessoa pode ser especialista em uma determinada área, mas principiante em outra. 

Podemos classificar um especialista como alguém que possui habilidades e estratégias bem desenvolvidas em determinada área de conhecimento. Em contrapartida, um principiante ainda não desenvolveu o suficiente determinadas habilidades e estratégias para resolver uma questão de determinada área. 

Em suma, o principiante terá mais dificuldade para resolver uma questão de uma área em que não é versado, diferentemente do especialista. \citeonline[p.17]{Echeverria1998} destacam que “a maior eficiência na solução de problemas pelos especialistas não seria devido a uma maior capacidade cognitiva geral e sim aos seus conhecimentos específicos”. 

Problemas de diferentes áreas de conhecimento necessitarão de diferentes estratégias e habilidades para sua resolução. Ou seja, o desenvolvimento dos mecanismos de resolução de um problema está diretamente ligado à área de conhecimento. Para os fins do presente trabalho, discutiremos sobre os processos de RP nas áreas da Matemática e CN. 
\section{PROBLEMAS EM MATEMÁTICA}

\citeonline[p.7]{Echeverria1998}, relembra que por muito tempo a ideia de “resolver problemas” no contexto escolar estava associada diretamente à Matemática. Essa relação entre Matemática e RP tomou forma em torno dos anos oitenta, época na qual os currículos de Matemática das escolas ocidentais traziam em seus objetivos tornar o aluno em uns “solucionadores de problemas”. 

Hoje, como já foi descrito nas seções anteriores, temos que a RP não é – ou pelo menos não deveria ser - um objeto de estudo apenas da Matemática, mas sim um elemento curricular comum a todas as disciplinas.  Como o intuito do presente trabalho é trabalhar a RP especificamente na área das Ciências da Natureza (CN), mais precisamente na disciplina de Física, é importante analisarmos como a RP se comporta na Matemática, pois, como veremos, existem pontos em que a RP da Matemática e das CN se misturam. 

Do ponto de vista de pesquisadores, podemos identificar dois pontos de interesse na RP em Matemática: o primeiro é a ideia de que o raciocínio utilizado na RP matemáticos pode refletir e estimular o raciocínio em outras áreas do conhecimento; o segundo: a ideia de que um aprofundamento nos conhecimentos e procedimentos da Matemática pode auxiliar no desenvolvimento de outras áreas científicas e tecnológicas \cite{Echeverria1998}. Estes dois pontos de vista acabam por definir a Matemática como uma área formal de ensino. Dessa forma, podemos dizer que ela possui procedimentos gerais para a RP e que, portanto, podem ser aplicados a diferente conteúdo. 

Quando analisamos o ponto de vista dos alunos a respeito da RP na Matemática, \citeonline[p.47]{Echeverria1998} dizem que: “os estudantes indicam que a Matemática e a solução de problemas matemáticos constituem um conhecimento descontextualizado cuja aprendizagem não possui outros objetivos a não ser o de obter boas notas na escola”. \citeonline{Schoenfeld1992} realizou uma revisão das ideias dos alunos em relação à cerca da natureza da Matemática, demonstrados na Tabela \ref{tab:pensamento_matematica}. 

\begin{table}[ht]
\caption{Pensamento dos estudantes a respeito da Matemática.}
\label{tab:pensamento_matematica}
\begin{tabularx}{\textwidth}{@{}X@{}}
\toprule
\textbf{Pensamentos dos estudantes} \\
\midrule
\begin{itemize}
    \item Os problemas matemáticos têm uma e somente uma resposta correta.
    \item Existe somente uma forma correta de resolver um problema matemático e, normalmente, o correto é seguir a última regra demonstrada em aula pelo professor.
    \item Os estudantes “normais” não são capazes de entender Matemática; somente podem esperar memorizá-la e aplicar mecanicamente aquilo que aprenderam sem entender.
    \item Os estudantes que entenderam Matemática devem ser capazes de resolver qualquer problema em cinco minutos ou menos.
    \item A Matemática ensinada na escola não tem nada a ver com o mundo real.
    \item As regras formais da Matemática são irrelevantes para os processos de descobrimento e de invenção.
\end{itemize} \\
\bottomrule
\end{tabularx}
\caption*{Fonte: Adaptado de \cite{Schoenfeld1992,Echeverria1998}.}
\end{table}

É perceptível, a partir da Tabela \ref{tab:pensamento_matematica}, que os alunos têm opinião contrária ao que é apresentado por pesquisadores a respeito da RP em Matemática. Para os estudantes, a RP parece ser apenas uma forma de cobrar um conteúdo que “não tem ligação com a vida real”. Ou seja, a ideia de que a RP deveria desenvolver capacidades e habilidades nos alunos, que auxiliariam em outras áreas do conhecimento, passa despercebida pelos estudantes. Nesse contexto, os alunos “nem sequer esperam chegar a compreender, em algum momento, os processos matemáticos que devem usar. Simplesmente esperam poder memorizá-los e aplicá-los mecanicamente, no momento oportuno” \cite{Echeverria1998}. 

Podemos destacar, segundo \citeonline{Schoenfeld1992}, que essas ideias dos estudantes estão ligadas com suas experiências em sala de aula, e essas experiências passam diretamente pela forma como o professor pensa e ensina a Matemática, e não pela forma como a disciplina está constituída. São estas ideias dos professores que refletem no “pensar matemático” dos alunos e refletem-nos diferentes significados conferidos à “solução de problemas”. 

\citeonline[p.14]{Echeverria1998}, relata que diversos autores chegaram a relatar até 14 distintos significados na utilização da expressão “solução de problemas” na Matemática. \citeonline{jane1983} resume esses significados em dois tipos: solução de problemas em Matemática equivale a qualquer tipo de atividade que precisa ser realizada, ou equivale a propor e tentar resolver uma questão com alto grau de dificuldade ou uma questão “inédita”. 

Quando se trata de sala de aula de Matemática, tradicionalmente, é aplicada a primeira definição de \citeonline{jane1983} para a RP. Ou seja, é comum considerarmos um problema matemático como sendo toda e qualquer atividade proposta. Contudo, como já fora discutido, sabemos que nem toda tarefa – independentemente de ser Matemática ou não pode ser considerado um problema. Utilizaremos aqui a distinção já discutida entre exercícios e problemas, ademais, separaremos os problemas matemáticos em dois tipos: quantitativos e qualitativos. 

Podemos dizer que, para que se possa resolver uma tarefa, o primeiro passo é a identificação da mesma como um exercício ou problema. Já discutimos as características dos exercícios neste capítulo e, por causa disso, nos atentaremos apenas aos exercícios matemáticos. 

\citeonline[p.14]{Echeverria1998} assinalam que, embora os currículos deem ênfase na solução de problemas, nas salas de aula o tempo é muito mais utilizado na resolução de exercícios do que de problemas. Essa análise é muito importante, pois, apesar de suas diferenças, ambas as tarefas, problemas e exercícios - geram diferentes consequências para a aprendizagem e correspondem a diferentes tipos de objetivos escolares. 

No caso dos exercícios é possível identificar dois tipos diferentes, segundo \citeonline[p.14]{Echeverria1998}: “O primeiro consiste na repetição de uma determinada técnica, previamente exposta pelo professor”. Ou seja, podemos identificar um dos tipos de exercícios matemáticos como sendo a replicação de uma técnica já aprendida. 

Ainda, segundo \citeonline[p.14]{Echeverria1998}: “o tipo de exercícios não pretende somente que seja automatizada uma série de técnicas, mas também que sejam aprendidos alguns procedimentos nos quais se inserem essas técnicas”. Dessa forma, esse segundo tipo de exercícios matemáticos se caracteriza por, além de aplicar as técnicas aprendidas previamente, fazer com que o aluno seja capaz de interpretar uma determinada situação e traduzi-la para o contexto matemático. 

Ao nos referirmos aos problemas matemáticos, conseguimos separá-los em problemas qualitativos e problemas quantitativos. Os problemas quantitativos são aqueles em que um resultado numérico é esperado, e ele, por si só, serve como resposta completa para a questão. Já os problemas qualitativos exigem dos alunos uma análise mais criteriosa da questão, onde a resposta numérica não é o fim, mas parte da solução. 

Após sermos capazes de classificar se uma tarefa é um problema ou um exercício, e de qual tipo, podemos começar a discutir o processo de ensino e aprendizagem da solução de um problema matemático. Para isso, precisamos lembrar que, segundo \citeonline{Polya2006}, a solução de problemas matemáticos se dá por meio de quatro etapas: compreensão, concepção de um plano, execução do plano e exame da solução alcançada. 

Segundo \citeonline{mayer1983}, podem reduzir os quatro passos de \citeonline{Polya2006} em apenas dois: tradução e solução do problema. Esses passos são importantes, pois, segundo \citeonline[p.20]{Echeverria1998}: “Nestes dois processos pode-se estabelecer uma correspondência com os três grandes eixos procedimentais estabelecidos nos currículos de Matemática: utilização de diferentes linguagens, utilização de algoritmos e utilização de habilidades”. 

Em resumo, ao utilizarmos a linguagem matemática para adequarmos o entendimento da questão e facilitar seu desenvolvimento estão traduzindo a linguagem escrita do problema para uma linguagem a qual sabemos como desenvolver uma possível solução. E a solução do problema é buscada, na nova linguagem já traduzida, através da aplicação de técnicas algorítmicas – como na solução de exercícios – unida ao desenvolvimento dos conhecimentos e estratégias necessárias para seu próprio desenvolvimento.

\section{PROBLEMAS EM CIÊNCIAS DA NATUREZA}

\citeonline{Pozo1998} assinalam como objetivo da formação científica - na Educação Básica - tornar os alunos capazes de enfrentar situações cotidianas, para que seja possível analisá-las e interpretá-las utilizando modelos conceituais e procedimentos científicos. O público estudado no presente trabalho são turmas de Graduação, mas os objetivos propostos por \citeonline{Pozo1998} estão de acordo com os princípios desejáveis para um estudante que conclua uma disciplina de Física Básica. 

É natural a inclusão de RP nas áreas científicas pois, segundo \citeonline{Pozo1998}: “Fora do formato acadêmico das atividades escolares, nossas perguntas ou inquietações sobre o funcionamento da natureza ou da tecnologia costumam aparecer sob a forma de problemas”. \citeonline{Claxton1991} sugere que “como consumidores de ciência que somos, precisamos ser capazes de resolver alguns dos problemas que o uso da ciência nos coloca”. 

Entretanto, de acordo com \citeonline{Pozo1998}: devemos reconhecer que a nossa capacidade – não só a de nossos alunos – de resolver problemas diários relacionados com a ciência e a tecnologia é bastante limitada, assim, podemos dizer que na maioria dos casos resolvemos os problemas cotidianos ligados à ciência através de procedimentos poucos ‘científicos’. Por exemplo, quando nos perguntamos se determinado carro é mais econômico que outro, estamos trabalhando com conceitos científicos dentro do dia a dia, mas não necessitamos fazer Ciência ou discuti-la em detalhes para resolver o problema. 

Se estivermos buscando que os alunos utilizem o que aprendem nas aulas de CN para resolver e decifrar problemas cotidianos, é necessário que seja dada mais importância para a RP dentro das disciplinas das áreas científicas. \citeonline{Pozo1998} defendem que: “se pretendemos que os alunos usem seus conhecimentos para resolver problemas, será necessário ensinar-lhes ciências resolvendo problemas”. 

Porém, temos uma dificuldade visível em sala de aula, onde, aparentemente, os problemas que são resolvidos nela são insuficientes ou ineficazes para garantir a desejada transferência do conhecimento acadêmico para o cotidiano. Para que consigamos entender melhor essa característica, devemos entender melhor os problemas escolares, científicos e cotidianos, apresentados na Tabela \ref{tab:problems}. 
\begin{comment}
\begin{center}
\begin{table}[t!]
\captionof{table}{Tipos de Problema.}
\label{tab:problems}
\begin{tabular}{|l|l|l|}
\hline 
\multicolumn{1}{|p{68.505936pt}}{\centering {\bfseries Tipos de Problemas}} & \multicolumn{1}{|p{192.72pt}}{\centering {\bfseries Caracter\'{\i}sticas }} & \multicolumn{1}{|p{113.67469pt}|}{\centering {\bfseries Exemplo }}\\ 
\hline 
\multicolumn{1}{|p{68.505936pt}}{\raggedright {\bfseries Escolares }} & \multicolumn{1}{|p{192.72pt}}{\raggedright S\~ao, geralmente, apresentados nos livros-texto. De acordo com \citeonline{Pozo1998}, tanto o projeto quanto o planejamento dos problemas escolares devem se basear no fato de que o conhecimento dos alunos est\'a mais pr\'oximo do conhecimento cotidiano do que do cient\'{\i}fico. Al\'em disso, \'e necess\'ario que os problemas escolares ajudem o estudante durante esta etapa de aquisi\c{c}\~ao de conhecimento. } & \multicolumn{1}{|p{113.67469pt}|}{\raggedright Se deixarmos uma bola rolar sobre a superf\'{\i}cie de uma mesa, ao atingir a borda ela cair\'a, alcan\c{c}ando o ch\~ao a certa dist\^ancia da mesa. Determine a rela\c{c}\~ao que existe entre a altura da mesa e a dist\^ancia percorrida pela bola antes de chegar ao ch\~ao \cite{Pozo1998}). }\\ 
\hline 
\multicolumn{1}{|p{68.505936pt}}{\raggedright {\bfseries Cient\'{\i}ficos }} & \multicolumn{1}{|p{192.72pt}}{\raggedright S\~ao aqueles que servem para fazer novas descobertas, podendo ser tanto cient\'{\i}ficas quanto tecnol\'ogicas. Suas solu\c{c}\~oes geram novos conhecimentos acerca do universo, objetos e sistemas nele contidos.} & \multicolumn{1}{|p{113.67469pt}|}{\raggedright [], como controlar com precis\~ao o movimento de um el\'etron dentro de um tubo de raios cat\'odicos de uma televis\~ao \cite{Pozo1998}. }\\ 
\hline 
\multicolumn{1}{|p{68.505936pt}}{\raggedright {\bfseries Cotidianos }} & \multicolumn{1}{|p{192.72pt}}{\raggedright S\~ao instigados por curiosidades ou d\'uvidas sobre o funcionamento das coisas ou sistemas no cotidiano. Diferentes pessoas podem discordar sobre se um determinado fen\^omeno \'e ou n\~ao um problema, baseado nas suas experi\^encias pessoais. } & \multicolumn{1}{|p{113.67469pt}|}{\raggedright A bola de basquete acertar\'a a cesta? \\ 
Por que os arco-\'{\i}ris surgem em dias chuvosos? \\ 
Passar com o carro sobre um buraco com maior velocidade altera a sensa\c{c}\~ao sentida pelo passageiro? }\\ 
\hline 
\end{tabular}
\caption*{Fonte: Construção do Autor.}
\end{table}
\end{center}
\end{comment}

\begin{table}[t!]
\caption{Tipos de Problema.}
\label{tab:problems}
\begin{tabular}{p{68.505936pt}|p{192.72pt}|p{113.67469pt}}
\hline 
\textbf{Tipos de Problemas} & \textbf{Características} & \textbf{Exemplo}\\ 
\hline 
\textbf{Escolares} & São, geralmente, apresentados nos livros-texto. De acordo com \citeonline{Pozo1998}, tanto o projeto quanto o planejamento dos problemas escolares devem se basear no fato de que o conhecimento dos alunos está mais próximo do conhecimento cotidiano do que do científico. Além disso, é necessário que os problemas escolares ajudem o estudante durante esta etapa de aquisição de conhecimento. & Se deixarmos uma bola rolar sobre a superfície de uma mesa, ao atingir a borda ela cairá, alcançando o chão a certa distância da mesa \cite{Pozo1998}. \\
\hline 
\textbf{Científicos} & São aqueles que servem para fazer novas descobertas, podendo ser tanto científicas quanto tecnológicas. Suas soluções geram novos conhecimentos acerca do universo, objetos e sistemas nele contidos. & Como controlar com precisão o movimento de um elétron dentro de um tubo de raios catódicos de uma televisão \cite{Pozo1998}. \\
\hline 
\textbf{Cotidianos} & São instigados por curiosidades ou dúvidas sobre o funcionamento das coisas ou sistemas no cotidiano. Diferentes pessoas podem discordar sobre se um determinado fenômeno é ou não um problema, baseado nas suas experiências pessoais. & A bola de basquete acertará a cesta? \\
& & Por que os arco-íris surgem em dias chuvosos? \\
& & Passar com o carro sobre um buraco com maior velocidade altera a sensação sentida pelo passageiro? \\
\hline 
\end{tabular}
\caption*{Fonte: Construção do Autor.}
\end{table}

Precisamos aqui focar nos problemas escolares, pois eles, em um grau superior de aplicação, serão utilizados no presente trabalho. Assim como separamos os problemas matemáticos na seção anterior, separaremos os problemas escolares nas CN em três tipos: problemas quantitativos, problemas qualitativos e pequenas pesquisas. 

Os problemas qualitativos são aqueles nos quais, segundo \citeonline{Pozo1998}, os alunos precisam resolver utilizando raciocínios teóricos, utilizando seus próprios conhecimentos, sem que haja necessidade de se embasar em resultados numéricos, além de não requererem para sua solução a realização ou manipulação de atividades experimentais. Ainda de acordo como os autores, esse tipo de problema acaba por ser aberto, de forma que seja necessário explicar ou predizer um fato, analisar situações cotidianas ou científicas e interpretá-las a partir dos conhecimentos pessoais ou de modelos conceituais. 

Sobre o ensino baseado em na resolução de problemas, \citeonline{Pozo1998} afirmam: “Em um ensino baseado na resolução de problemas, a tarefa do professor é geralmente mais complexa, diversificada e sutil do que num ensino expositivo tradicional, mas essa maior complexidade é notada especialmente no caso dos problemas qualitativos. ” 

Os problemas quantitativos são aqueles nos quais os alunos devem utilizar e manipular expressões e dados matemáticos a fim de chegar a uma solução, sendo ela numérica ou não. Um fato importante é de que os problemas quantitativos podem ou não ter um resultado também quantitativo, o que nos interessa nessa classificação é a forma como ele foi apresentado. Os problemas quantitativos permitem o estabelecimento de relações simples de diversas magnitudes científicas, o que acaba por facilitar a compreensão das leis da natureza. 

Os problemas quantitativos em CN vão ao encontro dos problemas matemáticos, pois nesses casos a fronteira entre o teor científico e o cálculo matemático não são bem definidas. \citeonline{Pozo1998} falam a respeito dessa característica: 

\begin{quoting}[leftmargin=4cm, rightmargin=0cm]
\noindent
Na verdade, é bastante comum observar que os alunos consideram ter resolvido um problema quando obtém um número (solução matemática), sem parar para pensar no significado desse número dentro do contexto científico no qual está enquadrado o problema (solução científica). O perigo é que as atividades matemáticas mascarem o problema da ciência, que o aluno, e às vezes o próprio professor, perceba e avalie o problema como uma tarefa essencialmente matemática. 
\end{quoting}

Por fim, temos as pequenas pesquisas, que são, de acordo com \citeonline{Pozo1998}, “aqueles trabalhos nos quais o aluno deve obter respostas para um problema por meio de um trabalho prático (tanto no laboratório escolar quanto fora dele) ”.  É importante entendermos que este tipo de problema é chamado de pequenas pesquisas, pois é uma aproximação do que consideramos pesquisa no âmbito científico. 
