\begin{resumo}[RESUMO]
\vspace{2em}
\begin{center}
%se der mais de uma página pode ser escrito em fonte 10pt
\textbf{\expandafter\MakeUppercase\expandafter{\imprimirtitulo}}\\
\vspace{1em}
AUTOR: \imprimirautor\\
ORIENTADOR: \imprimirorientador
\end{center}
\vspace{1em}
% texto
Este trabalho tem como foco o estudo da competência de Resolução de Problemas (RP) no contexto da área de Física, utilizando a metodologia de \textit{Scaffolding}. Para alcançar esse objetivo, adotamos a abordagem metodológica do \textit{Educational Design Research} para orientar nossas atividades de pesquisa. As aplicações em sala de aula foram realizadas por meio de dois modelos principais: Oficinas de Resolução de Problemas (ORP) e Atividades de Sala de Aula (ASA). O estudo contou com três aplicações de ORP em turmas do Ensino Superior, bem como três aplicações de ASA em turmas do Ensino Médio. As ORP e ASA foram complementares, interligando-se através de processos de \textit{redesign}, que foram gerados para atender às necessidades que surgiram ao longo das aplicações. Os objetivos da pesquisa estão claramente delineados, com enfoque na avaliação da eficácia da metodologia de \textit{Scaffolding} em relação à competência de Resolução de Problemas, tanto em cada aplicação específica como de forma geral.

\vspace{2em}
%no minimo tres separadas por .
\textbf{Palavras-chave}: Solução de Problemas. \textit{Feedback}. Sala de Aula. \textit{Redesign} Educacional. Ensino de Ciências.
\end{resumo}