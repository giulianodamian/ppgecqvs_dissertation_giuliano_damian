\begin{resumo}[ABSTRACT]
\vspace{2em}
\begin{center}
%se der mais de uma página pode ser escrito em fonte 10pt

\textbf{\expandafter\MakeUppercase\expandafter{\imprimirtituloestrangeiro}}\\
AUTHOR: \imprimirautor\\
ADVISOR: \imprimirorientador
\end{center}
\vspace{1em}
% texto
This dissertation focuses on studying the competence of Problem-Solving in the context of Physics, utilizing the methodology of Scaffolding. To achieve this objective, the Educational Design Research approach was adopted to guide the research activities. The applications in the classroom were carried out using two main models: Problem-Solving Workshops and Classroom Activities. The study included three applications of Problem-Solving Workshops in Higher Education classes and three applications of Classroom Activities in High School classes. The Problem-Solving Workshops and Classroom Activities were complementary, interconnected through redesign processes, which were generated to address emerging needs throughout the applications. The research objectives are clearly delineated, with a focus on evaluating the effectiveness of the Scaffolding methodology concerning the competence of Problem-Solving, both in each specific application and overall.

\vspace{2em}
%no minimo tres separadas por .
\textbf{Keywords}: Problem-Solving. Feedback. Classroom. Educational Redesign. Science Education.
\end{resumo}